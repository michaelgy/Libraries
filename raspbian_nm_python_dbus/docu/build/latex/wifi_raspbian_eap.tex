%% Generated by Sphinx.
\def\sphinxdocclass{report}
\documentclass[letterpaper,10pt,english]{sphinxhowto}
\ifdefined\pdfpxdimen
   \let\sphinxpxdimen\pdfpxdimen\else\newdimen\sphinxpxdimen
\fi \sphinxpxdimen=.75bp\relax

\PassOptionsToPackage{warn}{textcomp}
\usepackage[utf8]{inputenc}
\ifdefined\DeclareUnicodeCharacter
% support both utf8 and utf8x syntaxes
  \ifdefined\DeclareUnicodeCharacterAsOptional
    \def\sphinxDUC#1{\DeclareUnicodeCharacter{"#1}}
  \else
    \let\sphinxDUC\DeclareUnicodeCharacter
  \fi
  \sphinxDUC{00A0}{\nobreakspace}
  \sphinxDUC{2500}{\sphinxunichar{2500}}
  \sphinxDUC{2502}{\sphinxunichar{2502}}
  \sphinxDUC{2514}{\sphinxunichar{2514}}
  \sphinxDUC{251C}{\sphinxunichar{251C}}
  \sphinxDUC{2572}{\textbackslash}
\fi
\usepackage{cmap}
\usepackage[T1]{fontenc}
\usepackage{amsmath,amssymb,amstext}
\usepackage{babel}



\usepackage{times}
\expandafter\ifx\csname T@LGR\endcsname\relax
\else
% LGR was declared as font encoding
  \substitutefont{LGR}{\rmdefault}{cmr}
  \substitutefont{LGR}{\sfdefault}{cmss}
  \substitutefont{LGR}{\ttdefault}{cmtt}
\fi
\expandafter\ifx\csname T@X2\endcsname\relax
  \expandafter\ifx\csname T@T2A\endcsname\relax
  \else
  % T2A was declared as font encoding
    \substitutefont{T2A}{\rmdefault}{cmr}
    \substitutefont{T2A}{\sfdefault}{cmss}
    \substitutefont{T2A}{\ttdefault}{cmtt}
  \fi
\else
% X2 was declared as font encoding
  \substitutefont{X2}{\rmdefault}{cmr}
  \substitutefont{X2}{\sfdefault}{cmss}
  \substitutefont{X2}{\ttdefault}{cmtt}
\fi


\usepackage[Bjarne]{fncychap}
\usepackage{sphinx}

\fvset{fontsize=\small}
\usepackage{geometry}

% Include hyperref last.
\usepackage{hyperref}
% Fix anchor placement for figures with captions.
\usepackage{hypcap}% it must be loaded after hyperref.
% Set up styles of URL: it should be placed after hyperref.
\urlstyle{same}
\addto\captionsenglish{\renewcommand{\contentsname}{Contenido:}}

\usepackage{sphinxmessages}
\setcounter{tocdepth}{1}



\title{wifi\_raspbian\_eap}
\date{Nov 19, 2019}
\release{1.0}
\author{Michael Guerrero}
\newcommand{\sphinxlogo}{\vbox{}}
\renewcommand{\releasename}{Release}
\makeindex
\begin{document}

\pagestyle{empty}
\sphinxmaketitle
\pagestyle{plain}
\sphinxtableofcontents
\pagestyle{normal}
\phantomsection\label{\detokenize{index::doc}}


Esta libreria es desarrollada utilizando \sphinxstyleemphasis{Python3} y la interfaz para \sphinxstyleemphasis{dbus} de \sphinxstyleemphasis{NetworkManager},
lo cual permite usar el servicio \sphinxstyleemphasis{NetworkManager}, que es un administrador de redes muy popular
y robusto en sistemas Linux, a través de \sphinxstyleemphasis{Python}.

En \sphinxstyleemphasis{Raspbian} por defecto se encuentra el servicio y la interfaz grafica del administrador de
conexiones \sphinxstyleemphasis{dhcpcd}, este es muy intuitivo y provee unas funcionalidades muy básicas, pero esto
a su vez impide que sea posible establecer conexiones que requieran una configuración específica,
como lo son las conexiones \sphinxstylestrong{WPA2-EAP}. Por esta razón y por la necesidad de configurar el
administrador de conexiones desde \sphinxstyleemphasis{python}, se ha desarrollado esta libreria, que pone en
disposición funciones de fácil uso y configuración, especialmente para conexiones:
\begin{itemize}
\item {} 
Sin seguridad (abiertas).

\item {} 
Con protocolo de seguridad \sphinxstylestrong{TKIP-CCMP} (contraseña).

\item {} 
Con protocolo de seguridad \sphinxstylestrong{EAP} (usuario y contraseña).

\end{itemize}


\chapter{Instalación}
\label{\detokenize{index:instalacion}}

\section{Deshabilitar dhcpcd (servicio e icono en el panel o barra de herramientas)}
\label{\detokenize{index:deshabilitar-dhcpcd-servicio-e-icono-en-el-panel-o-barra-de-herramientas}}
Es necesario deshabilitar el servicio \sphinxstyleemphasis{dhcpcd} que se inicia en el proceso de arranque, para
evitar tener conflictos con el servicio de \sphinxstyleemphasis{NetworkManager}, esto logra ejecutando en consola
el siguiente comando:

\begin{sphinxVerbatim}[commandchars=\\\{\}]
\PYG{n}{sudo} \PYG{n}{sytemctl} \PYG{n}{disable} \PYG{n}{dhcpcd}
\end{sphinxVerbatim}

Además en el panel aparece un icono relacionado con la interfaz grafica de \sphinxstyleemphasis{dhcpcd}, este icono se puede
deshabilitar comentando del archivo \sphinxstylestrong{TODO} y las siguientes lineas:

\begin{sphinxVerbatim}[commandchars=\\\{\}]
\PYG{o}{*}\PYG{o}{*}\PYG{n}{TODO}\PYG{o}{*}\PYG{o}{*}
\end{sphinxVerbatim}


\section{Instalar NetworkManager y las libreras necesarias para python}
\label{\detokenize{index:instalar-networkmanager-y-las-libreras-necesarias-para-python}}
Para instalar \sphinxstyleemphasis{NetworkManger} se debe ejecutar el siguiente comando en consola:

\begin{sphinxVerbatim}[commandchars=\\\{\}]
\PYG{n}{sudo} \PYG{n}{apt}\PYG{o}{\PYGZhy{}}\PYG{n}{get} \PYG{n}{install} \PYG{n}{network}\PYG{o}{\PYGZhy{}}\PYG{n}{manager}
\end{sphinxVerbatim}

opcionalmente se puede instalar un inspeccionador de la interaz \sphinxstyleemphasis{dbus}
con el siguiente comando:

\begin{sphinxVerbatim}[commandchars=\\\{\}]
\PYG{n}{sudo} \PYG{n}{apt}\PYG{o}{\PYGZhy{}}\PYG{n}{get} \PYG{n}{install} \PYG{n}{d}\PYG{o}{\PYGZhy{}}\PYG{n}{feet}
\end{sphinxVerbatim}

Se deben instalar las dependencias de python necesarias para controlar la interfaz \sphinxstyleemphasis{dbus}:

\begin{sphinxVerbatim}[commandchars=\\\{\}]
\PYG{n}{sudo} \PYG{n}{apt} \PYG{n}{install} \PYG{n}{python3}\PYG{o}{\PYGZhy{}}\PYG{n}{gi} \PYG{n}{python3}\PYG{o}{\PYGZhy{}}\PYG{n}{gi}\PYG{o}{\PYGZhy{}}\PYG{n}{cairo} \PYG{n}{gir1}\PYG{o}{.}\PYG{l+m+mi}{2}\PYG{o}{\PYGZhy{}}\PYG{n}{gtk}\PYG{o}{\PYGZhy{}}\PYG{l+m+mf}{3.0}
\end{sphinxVerbatim}

Y por ultimo se instala la interfaz de comunicación de \sphinxstyleemphasis{NetworkManager} para \sphinxstyleemphasis{dbus}:

\begin{sphinxVerbatim}[commandchars=\\\{\}]
\PYG{n}{sudo} \PYG{n}{pip3} \PYG{n}{install} \PYG{n}{python}\PYG{o}{\PYGZhy{}}\PYG{n}{networkmanager}
\end{sphinxVerbatim}


\chapter{Uso}
\label{\detokenize{index:uso}}

\section{Impotar la libreria}
\label{\detokenize{index:impotar-la-libreria}}
Para utilizar esta libreria primero debe importarla:

\begin{sphinxVerbatim}[commandchars=\\\{\}]
\PYG{k+kn}{import} \PYG{n+nn}{nm\PYGZus{}dbus\PYGZus{}python}
\end{sphinxVerbatim}

Debe asegurarse de que la carpeta nm\_dbus\_python se encuentra en el directorio de trabajo o en el Path.


\section{Conexion a redes Wifi:}
\label{\detokenize{index:conexion-a-redes-wifi}}
Para conectarse a una red Wifi sin seguridad, puede utilizar el siguiente comando:

\begin{sphinxVerbatim}[commandchars=\\\{\}]
\PYG{n}{nm\PYGZus{}dbus\PYGZus{}python}\PYG{o}{.}\PYG{n}{connecttoAP}\PYG{p}{(}\PYG{n}{ssid}\PYG{p}{)}
\end{sphinxVerbatim}

Para conectarse a una red Wifi con seguridad PSK, puede utilizar el siguiente comando:

\begin{sphinxVerbatim}[commandchars=\\\{\}]
\PYG{n}{nm\PYGZus{}dbus\PYGZus{}python}\PYG{o}{.}\PYG{n}{connecttoAP}\PYG{p}{(}\PYG{n}{ssid}\PYG{p}{,} \PYG{n}{password}\PYG{o}{=}\PYG{n}{pssw}\PYG{p}{)}
\end{sphinxVerbatim}

Para conectarse a una red Wifi con seguridad EAP, puede utilizar el siguiente comando:

\begin{sphinxVerbatim}[commandchars=\\\{\}]
\PYG{n}{nm\PYGZus{}dbus\PYGZus{}python}\PYG{o}{.}\PYG{n}{connecttoAP}\PYG{p}{(}\PYG{n}{ssid}\PYG{p}{,} \PYG{n}{password}\PYG{o}{=}\PYG{n}{pssw}\PYG{p}{,} \PYG{n}{user}\PYG{o}{=}\PYG{n}{usr}\PYG{p}{)}
\end{sphinxVerbatim}

en los ejemplos anteriores, los parametros ssid, pssw y usr, son cadenas de caracteres,
por ejemplo:

\begin{sphinxVerbatim}[commandchars=\\\{\}]
\PYG{n}{nm\PYGZus{}dbus\PYGZus{}python}\PYG{o}{.}\PYG{n}{connecttoAP}\PYG{p}{(}\PYG{l+s+s2}{\PYGZdq{}}\PYG{l+s+s2}{wifi network}\PYG{l+s+s2}{\PYGZdq{}}\PYG{p}{,} \PYG{n}{password}\PYG{o}{=}\PYG{l+s+s2}{\PYGZdq{}}\PYG{l+s+s2}{password}\PYG{l+s+s2}{\PYGZdq{}}\PYG{p}{,} \PYG{n}{user}\PYG{o}{=}\PYG{l+s+s2}{\PYGZdq{}}\PYG{l+s+s2}{user}\PYG{l+s+s2}{\PYGZdq{}}\PYG{p}{)}
\end{sphinxVerbatim}

ademas, la funcion connectoAP cuenta con un parametro adicional para establecer
un tiempo maximo de espera para la respuesta de la conexion, por ejemplo, si se
va a esperar como maximo 5 segundos a que se conecte a una red, se debe utilizar
de la siguiente forma:

\begin{sphinxVerbatim}[commandchars=\\\{\}]
\PYG{n}{nm\PYGZus{}dbus\PYGZus{}python}\PYG{o}{.}\PYG{n}{connecttoAP}\PYG{p}{(}\PYG{l+s+s2}{\PYGZdq{}}\PYG{l+s+s2}{wifi network}\PYG{l+s+s2}{\PYGZdq{}}\PYG{p}{,} \PYG{n}{password}\PYG{o}{=}\PYG{l+s+s2}{\PYGZdq{}}\PYG{l+s+s2}{password}\PYG{l+s+s2}{\PYGZdq{}}\PYG{p}{,} \PYG{n}{timeout}\PYG{o}{=}\PYG{l+m+mi}{5}\PYG{p}{)}
\end{sphinxVerbatim}

si se conoce el bssid (direccion MAC) de la red se puede utilizar el siguiente
comando:

\begin{sphinxVerbatim}[commandchars=\\\{\}]
\PYG{n}{nm\PYGZus{}dbus\PYGZus{}python}\PYG{o}{.}\PYG{n}{connecttoAP}\PYG{p}{(}\PYG{l+s+s2}{\PYGZdq{}}\PYG{l+s+s2}{\PYGZdq{}}\PYG{p}{,}\PYG{n}{bssid}\PYG{o}{=}\PYG{l+s+s2}{\PYGZdq{}}\PYG{l+s+s2}{CC:35:40:98:A8:3F}\PYG{l+s+s2}{\PYGZdq{}}\PYG{p}{,} \PYG{n}{password}\PYG{o}{=}\PYG{l+s+s2}{\PYGZdq{}}\PYG{l+s+s2}{password}\PYG{l+s+s2}{\PYGZdq{}}\PYG{p}{)}
\end{sphinxVerbatim}

tambien se puede ingresar el bssid y el ssid de la red:

\begin{sphinxVerbatim}[commandchars=\\\{\}]
\PYG{n}{nm\PYGZus{}dbus\PYGZus{}python}\PYG{o}{.}\PYG{n}{connecttoAP}\PYG{p}{(}\PYG{l+s+s2}{\PYGZdq{}}\PYG{l+s+s2}{wifi network}\PYG{l+s+s2}{\PYGZdq{}}\PYG{p}{,}\PYG{n}{bssid}\PYG{o}{=}\PYG{l+s+s2}{\PYGZdq{}}\PYG{l+s+s2}{CC:35:40:98:A8:3F}\PYG{l+s+s2}{\PYGZdq{}}\PYG{p}{,} \PYG{n}{password}\PYG{o}{=}\PYG{l+s+s2}{\PYGZdq{}}\PYG{l+s+s2}{password}\PYG{l+s+s2}{\PYGZdq{}}\PYG{p}{)}
\end{sphinxVerbatim}

en este caso se tratara de conectar a la red “wifi network”, en caso de encontrar
la red o de no poder acceder a ella, intentara conectarse a la red con bssid = \sphinxcode{\sphinxupquote{"CC:35:40:98:A8:3F"}}


\chapter{API}
\label{\detokenize{index:module-nm_dbus_python}}\label{\detokenize{index:api}}\index{nm\_dbus\_python (module)@\spxentry{nm\_dbus\_python}\spxextra{module}}\index{accessPoints() (in module nm\_dbus\_python)@\spxentry{accessPoints()}\spxextra{in module nm\_dbus\_python}}

\begin{fulllineitems}
\phantomsection\label{\detokenize{index:nm_dbus_python.accessPoints}}\pysiglinewithargsret{\sphinxcode{\sphinxupquote{nm\_dbus\_python.}}\sphinxbfcode{\sphinxupquote{accessPoints}}}{\emph{prop}, \emph{value}}{}
Function used to get available access points.
\begin{quote}\begin{description}
\item[{Parameters}] \leavevmode\begin{itemize}
\item {} 
\sphinxstyleliteralstrong{\sphinxupquote{prop}} (\sphinxstyleliteralemphasis{\sphinxupquote{str}}) \textendash{} is the property that must have an access point of interest

\item {} 
\sphinxstyleliteralstrong{\sphinxupquote{value}} (\sphinxstyleliteralemphasis{\sphinxupquote{str}}) \textendash{} is the respective value that property must have

\end{itemize}

\item[{Returns}] \leavevmode
A tuple with availables access points and the wireless device

\item[{Return type}] \leavevmode
tuple

\end{description}\end{quote}

\end{fulllineitems}

\index{connectProfile() (in module nm\_dbus\_python)@\spxentry{connectProfile()}\spxextra{in module nm\_dbus\_python}}

\begin{fulllineitems}
\phantomsection\label{\detokenize{index:nm_dbus_python.connectProfile}}\pysiglinewithargsret{\sphinxcode{\sphinxupquote{nm\_dbus\_python.}}\sphinxbfcode{\sphinxupquote{connectProfile}}}{\emph{setting}, \emph{key}, \emph{value}}{}
Function used to get previously saved connection profiles.
\begin{quote}\begin{description}
\item[{Parameters}] \leavevmode\begin{itemize}
\item {} 
\sphinxstyleliteralstrong{\sphinxupquote{setting}} (\sphinxstyleliteralemphasis{\sphinxupquote{str}}) \textendash{} is the setting that must have a connection profile

\item {} 
\sphinxstyleliteralstrong{\sphinxupquote{key}} (\sphinxstyleliteralemphasis{\sphinxupquote{str}}) \textendash{} is the respective key that a property of setting must have

\item {} 
\sphinxstyleliteralstrong{\sphinxupquote{value}} (\sphinxstyleliteralemphasis{\sphinxupquote{str}}) \textendash{} is the respective value of key must have

\end{itemize}

\item[{Returns}] \leavevmode
A settings object

\end{description}\end{quote}

\end{fulllineitems}

\index{deactive\_wireless\_conn() (in module nm\_dbus\_python)@\spxentry{deactive\_wireless\_conn()}\spxextra{in module nm\_dbus\_python}}

\begin{fulllineitems}
\phantomsection\label{\detokenize{index:nm_dbus_python.deactive_wireless_conn}}\pysiglinewithargsret{\sphinxcode{\sphinxupquote{nm\_dbus\_python.}}\sphinxbfcode{\sphinxupquote{deactive\_wireless\_conn}}}{}{}
Function used to disconnect all active wireless connections.

\end{fulllineitems}

\index{get\_all\_ap\_info() (in module nm\_dbus\_python)@\spxentry{get\_all\_ap\_info()}\spxextra{in module nm\_dbus\_python}}

\begin{fulllineitems}
\phantomsection\label{\detokenize{index:nm_dbus_python.get_all_ap_info}}\pysiglinewithargsret{\sphinxcode{\sphinxupquote{nm\_dbus\_python.}}\sphinxbfcode{\sphinxupquote{get\_all\_ap\_info}}}{}{}
Function used to get the attributes of interest of the available networks.
\begin{quote}\begin{description}
\item[{Returns}] \leavevmode
a list of dictionaries with the attributes in attr\_interest list

\item[{Return type}] \leavevmode
list

\end{description}\end{quote}

\end{fulllineitems}

\index{create\_connProfile() (in module nm\_dbus\_python)@\spxentry{create\_connProfile()}\spxextra{in module nm\_dbus\_python}}

\begin{fulllineitems}
\phantomsection\label{\detokenize{index:nm_dbus_python.create_connProfile}}\pysiglinewithargsret{\sphinxcode{\sphinxupquote{nm\_dbus\_python.}}\sphinxbfcode{\sphinxupquote{create\_connProfile}}}{\emph{ssid}, \emph{bssid=''}, \emph{user=''}, \emph{password=''}}{}
Function used to create a new connection profile.
\begin{quote}\begin{description}
\item[{Parameters}] \leavevmode
\sphinxstyleliteralstrong{\sphinxupquote{ssid}} (\sphinxstyleliteralemphasis{\sphinxupquote{str}}) \textendash{} is the ssid of the access point.

\item[{Returns}] \leavevmode
A dictionary with all the information necessary to create a connection profile

\item[{Return type}] \leavevmode
dict

\end{description}\end{quote}

\end{fulllineitems}

\index{connecttoAP() (in module nm\_dbus\_python)@\spxentry{connecttoAP()}\spxextra{in module nm\_dbus\_python}}

\begin{fulllineitems}
\phantomsection\label{\detokenize{index:nm_dbus_python.connecttoAP}}\pysiglinewithargsret{\sphinxcode{\sphinxupquote{nm\_dbus\_python.}}\sphinxbfcode{\sphinxupquote{connecttoAP}}}{\emph{ssid}, \emph{bssid=''}, \emph{user=''}, \emph{password=''}, \emph{timeout=-1}}{}
This functions try to connect to a network with the given parameters

Currently this function can only connecto to:
\begin{itemize}
\item {} 
open networks

\item {} 
psk tkip networks

\item {} 
peap mschapv2 networks

\end{itemize}

this function prioritizes ssid over bssid
always return the bssid of the network connected to
\begin{quote}\begin{description}
\item[{Parameters}] \leavevmode\begin{itemize}
\item {} 
\sphinxstyleliteralstrong{\sphinxupquote{ssid}} (\sphinxstyleliteralemphasis{\sphinxupquote{str}}) \textendash{} is the ssid of the access point to connect to

\item {} 
\sphinxstyleliteralstrong{\sphinxupquote{timeout}} (\sphinxstyleliteralemphasis{\sphinxupquote{int}}) \textendash{} is the numbers of seconds to wait for connection, -1 for wait until connect

\end{itemize}

\item[{Returns}] \leavevmode
False if the connection failed or a string with the MacAddress

\item[{Return type}] \leavevmode
bool\textbar{}str

\end{description}\end{quote}

\end{fulllineitems}

\index{connectMAC() (in module nm\_dbus\_python)@\spxentry{connectMAC()}\spxextra{in module nm\_dbus\_python}}

\begin{fulllineitems}
\phantomsection\label{\detokenize{index:nm_dbus_python.connectMAC}}\pysiglinewithargsret{\sphinxcode{\sphinxupquote{nm\_dbus\_python.}}\sphinxbfcode{\sphinxupquote{connectMAC}}}{\emph{bssid}, \emph{user=''}, \emph{password=''}}{}
This functions try to connect to a network with the given bssid(MacAddress)

Currently this function can only connecto to:
\begin{itemize}
\item {} 
open networks

\item {} 
psk tkip networks

\item {} 
peap mschapv2 networks

\end{itemize}

this function prioritizes ssid over bssid
always return the bssid of the network connected to
\begin{quote}\begin{description}
\item[{Parameters}] \leavevmode
\sphinxstyleliteralstrong{\sphinxupquote{bssid}} (\sphinxstyleliteralemphasis{\sphinxupquote{str}}) \textendash{} is the bssid(MacAddress) of the access point to connect to

\item[{Returns}] \leavevmode
False if the connection failed or a string with the MacAddress

\item[{Return type}] \leavevmode
bool\textbar{}str

\end{description}\end{quote}

\end{fulllineitems}

\index{JaverianaCali() (in module nm\_dbus\_python)@\spxentry{JaverianaCali()}\spxextra{in module nm\_dbus\_python}}

\begin{fulllineitems}
\phantomsection\label{\detokenize{index:nm_dbus_python.JaverianaCali}}\pysiglinewithargsret{\sphinxcode{\sphinxupquote{nm\_dbus\_python.}}\sphinxbfcode{\sphinxupquote{JaverianaCali}}}{\emph{user}, \emph{password}}{}
Function used to connect to the network JaverianaCali with the given user and password.

Is assumed that JaverianaCali is an 802.1x connection. This function is only for test, 
use connecttoAP instead.
\begin{quote}\begin{description}
\item[{Parameters}] \leavevmode\begin{itemize}
\item {} 
\sphinxstyleliteralstrong{\sphinxupquote{user}} (\sphinxstyleliteralemphasis{\sphinxupquote{str}}) \textendash{} is a string with a valid user

\item {} 
\sphinxstyleliteralstrong{\sphinxupquote{password}} (\sphinxstyleliteralemphasis{\sphinxupquote{str}}) \textendash{} is a string with the password of the user

\end{itemize}

\end{description}\end{quote}

\end{fulllineitems}

\index{Invitados\_JaverianaCali() (in module nm\_dbus\_python)@\spxentry{Invitados\_JaverianaCali()}\spxextra{in module nm\_dbus\_python}}

\begin{fulllineitems}
\phantomsection\label{\detokenize{index:nm_dbus_python.Invitados_JaverianaCali}}\pysiglinewithargsret{\sphinxcode{\sphinxupquote{nm\_dbus\_python.}}\sphinxbfcode{\sphinxupquote{Invitados\_JaverianaCali}}}{}{}
Function used to connect to the network ‘Invitados JaverianaCali’.

Is assumed that Invitados JaverianaCali is open. This function is only for test, 
use connecttoAP instead.
\begin{quote}\begin{description}
\item[{Parameters}] \leavevmode\begin{itemize}
\item {} 
\sphinxstyleliteralstrong{\sphinxupquote{user}} (\sphinxstyleliteralemphasis{\sphinxupquote{str}}) \textendash{} is a string with a valid user

\item {} 
\sphinxstyleliteralstrong{\sphinxupquote{password}} (\sphinxstyleliteralemphasis{\sphinxupquote{str}}) \textendash{} is a string with the password of the user

\end{itemize}

\end{description}\end{quote}

\end{fulllineitems}



\renewcommand{\indexname}{Python Module Index}
\begin{sphinxtheindex}
\let\bigletter\sphinxstyleindexlettergroup
\bigletter{n}
\item\relax\sphinxstyleindexentry{nm\_dbus\_python}\sphinxstyleindexpageref{index:\detokenize{module-nm_dbus_python}}
\end{sphinxtheindex}

\renewcommand{\indexname}{Index}
\printindex
\end{document}